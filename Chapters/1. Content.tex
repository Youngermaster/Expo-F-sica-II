\section{Introducción}

En el campo de la medicina de emergencia, el desfibrilador representa uno de los avances tecnológicos más significativos, salvando innumerables vidas a través de su capacidad para restaurar un ritmo cardíaco normal en pacientes que experimentan arritmias potencialmente mortales. Sin embargo, el funcionamiento y la eficacia de estos dispositivos trascienden la mera aplicación médica, encontrando sus raíces en principios físicos fundamentales.En el presente documento se explicara y describira los temas a tratar en esta exposición e investigación grupal acerca del desfibrilador, su importancia en la medicina y su funcionamiento físico, tratando conceptos importantes de física y así también dando a conocer estas aplicaciones y su importancia en el mundo actual, y lo que nos espera para el futuro de la medicina y la física. Este documento busca explorar el desfibrilador no solo como un instrumento médico, sino como una aplicación práctica de la física, desentrañando los mecanismos que permiten a este dispositivo intervenir de manera tan crítica en los momentos de mayor necesidad.

\section{Objetivos}

\begin{itemize}
    \item Explicar qué es un desfibrilador y cómo se relaciona con conceptos físicos fundamentales.
    \item Examinar los componentes de un desfibrilador y los procesos físicos que permiten su funcionamiento.
    \item Establecer la conexión entre los principios de la física tratados en clase y la operación de las baterías y capacitores dentro de un desfibrilador.
    \item Demostrar mediante un experimento sencillo el concepto de la descarga de un capacitor.
\end{itemize}

Además, es fundamental comprender la importancia de exponer y discutir temas como el desfibrilador y su relación con la física médica, la bioelectricidad y la ingeniería biomédica, especialmente en el contexto educativo y para el público en general en la actualidad. En primer lugar, brindar conocimientos sobre estos temas permite a los estudiantes y a la sociedad en general entender cómo la física y la tecnología se aplican en el campo de la medicina, ampliando su comprensión sobre el funcionamiento del cuerpo humano y las herramientas que se utilizan para preservar la salud y salvar vidas. Además, la exposición de estos temas promueve la conciencia sobre la importancia de la innovación científica y tecnológica en el ámbito médico, incentivando el interés por carreras y disciplinas relacionadas con la investigación y el desarrollo de dispositivos médicos. Al comprender cómo la física y la tecnología están involucradas en la atención médica, los estudiantes pueden verse inspirados a contribuir con nuevas ideas y avances que mejoren la calidad de vida de las personas y abordan desafíos de salud global. de la misma manera, familiarizarse con temas como la bioelectricidad y la historia del desfibrilador permite a la sociedad entender mejor cómo ha evolucionado la medicina a lo largo del tiempo y cómo los avances científicos han impactado positivamente en la atención médica y la supervivencia de los pacientes. Esto fomenta una apreciación más profunda de la ciencia y la tecnología y su papel en la sociedad moderna, promoviendo una cultura de innovación, prevención y cuidado de la salud.

\newpage
\section{Descripción de la exposición}

La exposición se centrará en el desfibrilador como un punto de convergencia entre la física médica, la bioelectricidad y la ingeniería biomédica, buscando trazar un recorrido desde los fundamentos teóricos hasta las aplicaciones prácticas. Empezaremos explorando los principios de la física que tienen aplicaciones directas en la medicina, revelando cómo ha impulsado el desarrollo de tecnologías médicas vitales.
\newline
Profundizaremos en la fisiología del corazón, abordando su naturaleza eléctrica y las consecuencias de las arritmias cardíacas, enfatizando el papel crítico de los impulsos eléctricos y su regulación. Al introducir la bioelectricidad, conectaremos estos conceptos con el origen de los potenciales de acción cardíaca y su importancia en la función saludable del corazón.\newline
La historia y la evolución tecnológica del desfibrilador nos permitirán apreciar su impacto trascendental en la medicina, proporcionando estadísticas de éxito en la reanimación y cómo ha contribuido a la reducción de la mortalidad cardíaca. Los procesos físicos serán desglosados para explicar el funcionamiento del desfibrilador, destacando la importancia de la relación entre potencial eléctrico y corriente. \newline \hfill \break
Nos adentraremos en los componentes clave del desfibrilador, examinando cómo los capacitores, inductores y baterías juegan roles esenciales en su operación. Discutiremos los pasos críticos en la desfibrilación, subrayando cómo la precisión y rapidez son aseguradas por los fundamentos de la física. \newline \hfill \break
Exploramos los distintos tipos de desfibriladores, desde los DEA hasta los implantables, y cómo cada uno se ajusta a diferentes contextos y necesidades. La física detrás del desfibrilador será analizada a través del estudio de circuitos y la interacción de la energía con el tejido cardíaco. \newline \hfill \break
La seguridad y eficacia serán temas cruciales, donde discutiremos el diseño, las características de seguridad y los criterios para pruebas clínicas. Miraremos hacia el futuro, contemplando las innovaciones tecnológicas y las posibles direcciones en la investigación y el desarrollo. \newline \hfill \break
Las simulaciones y el modelado se presentarán como herramientas clave para comprender la dinámica cardíaca y mejorar la desfibrilación, tanto en la investigación como en la capacitación. Extendiendo el alcance de los desfibriladores, analizaremos su presencia y relevancia en diversos entornos, así como los programas de concienciación y capacitación. \newline \hfill \break
Finalmente, reflexionaremos sobre los aspectos éticos y legales asociados al uso de desfibriladores, considerando las regulaciones y estándares que rigen para fabricantes y usuarios, concluyendo con una visión integral de su importancia en la medicina de emergencia y su potencial para salvar vidas.

\section{Temas a tratar}
La siguiente es una lista de los temas a tratar en la exposición sobre el desfibrilador y una breve descripción de lo que se hablara.

\subsection{Física médica:}
\begin{itemize}
    \item Principios de la física aplicados en la medicina.
    \item El papel de la física en el desarrollo de tecnologías médicas.
\end{itemize}

La Física Médica es la aplicación de los principios de la física a la medicina o al cuidado de la salud. Básicamente, es una forma de utilizar nuestro conocimiento de la física para desarrollar herramientas y tratamientos que ayuden a los humanos a vivir más tiempo y ser más saludables.Los principios de la física médica pueden incluir las teorías asociadas con amplitudes, presión de fluido, frecuencias y ondas. Las aplicaciones de estos principios se pueden encontrar en radiología de diagnóstico, medicina nuclear y oncología radioterápica. Además la física ha tenido una gran influencia en la imagenología médica. La radiología, la tomografía y la resonancia magnética son solo algunas de las técnicas que se utilizan en la actualidad para obtener imágenes del interior del cuerpo humano. Todas estas técnicas se basan en principios físicos como la absorción de radiación por los tejidos, la utilización de campos magnéticos y la emisión de ondas sonoras. Aparte de la imagenología, la física también ha tenido importantes aportaciones en la terapia médica. La radioterapia, por ejemplo, utiliza radiación ionizante para destruir células cancerosas. Otro ejemplo es la terapia con láser, que utiliza la luz para tratar diferentes enfermedades. La Física Médica también desempeña una importante función en la prevención de enfermedades, la investigación biológica y médica, y en la optimización de ciertas actividades sanitarias.

\subsection{Fisiología del Corazón:}
\begin{itemize}
    \item Cómo la electricidad controla el ritmo cardíaco.
    \item Qué sucede durante una arritmia cardíaca.
\end{itemize}

El corazón es un órgano muscular hueco que funciona como una bomba, impulsando la sangre a través de las arterias para distribuirla por todo nuestro organismo. Su tamaño es similar al de un puño y su peso se sitúa entre los 250 y los 350 gramos. El corazón está dividido en cuatro cavidades: dos superiores llamadas aurículas y dos inferiores llamadas ventrículos. La aurícula y el ventrículo derecho forman lo que se conoce como el corazón derecho, mientras que la aurícula y el ventrículo izquierdo forman el corazón izquierdo.El corazón tiene un sistema eléctrico especial llamado el sistema de conducción cardíaca. Este sistema controla la frecuencia y el ritmo de los latidos. Con cada latido, una señal eléctrica recorre el corazón desde la parte superior hasta la inferior. Las células del nódulo sinoatrial (SA) en la parte superior del corazón se conocen como el marcapasos del corazón porque la velocidad a la que estas células envían señales eléctricas determina la velocidad a la que late todo el corazón (frecuencia cardíaca). Las señales eléctricas producen la contracción de los músculos. El sistema de conducción eléctrica del corazón tiene la función principal de permitir que la sangre que bombea el corazón se distribuya por todo el cuerpo (es decir, sea bombeada por todo el organismo). Coordina las contracciones de las cavidades del corazón para que este lata de forma correcta.

\subsection{Bioelectricidad:}
\begin{itemize}
    \item Conceptos de bioelectricidad y señales eléctricas en el cuerpo humano.
    \item El origen del potencial de acción cardíaca.
\end{itemize}

La bioelectricidad es el estudio de los flujos eléctricos producidos dentro de los cuerpos de los seres vivos en diferentes porciones y con funciones particulares. Los nervios, por ejemplo, transmiten comandos de acción y estímulo del sistema nervioso a todos los demás sistemas a través de señales bioeléctricas que son esenciales para mantener la salud y la homeostasis.El cuerpo humano es un complejo sistema biológico en el que la electricidad desempeña un papel fundamental. Desde el latido del corazón hasta la transmisión de señales nerviosas, todas las funciones vitales dependen de la presencia de corriente eléctrica en nuestro organismo. El potencial de acción cardíaco corresponde a una rápida despolarización de la membrana, seguida de la repolarización hasta el potencial de membrana en reposo en la fibra miocárdica. Este potencial se origina en el nodo sinoauricular, y se propaga por todo el músculo cardíaco a través del. Los impulsos cardíacos que generan la contracción miocárdica tienen su origen en el nodo sinoauricular, el cual se ubica en la aurícula derecha, donde se conduce el potencial de acción a través de los haces internodales hasta llegar al nodo auriculoventricular, de donde se propaga el potencial de acción por las fibras de Purkinje a toda la superficie ventricular.

\subsection{Introducción al desfibrilador:}
\begin{itemize}
    \item Descripción general de un desfibrilador.
    \item Sus partes principales y funciones.
\end{itemize}

Un desfibrilador es un dispositivo médico que permite la aplicación de descargas eléctricas para lograr el restablecimiento del ritmo cardíaco normal. Este dispositivo es capaz de detectar alteraciones en el ritmo cardíaco y administrar una descarga eléctrica al corazón cuando sea necesario. La descarga tiene la capacidad de restablecer el ritmo ‘sinusal’, es decir, el ritmo cardíaco correcto coordinado por el marcapasos natural del corazón.

\subsection{Partes de un desfibrilador:}
\begin{itemize}
    \item Electrodos: Son los encargados de transmitir la descarga eléctrica al corazón del paciente.
    \item Cable de conexión: Conecta los electrodos con el desfibrilador.
    \item Batería: Proporciona la energía necesaria para que el desfibrilador funcione.
    \item Pantalla: Muestra órdenes visuales que sirven de guía y facilitan el manejo del desfibrilador.
    \item Botón de carga: Prepara el desfibrilador para administrar la descarga.
    \item Botón de choque: Administra la descarga eléctrica.
    \item Almacenamiento de datos: Almacena los eventos relacionados con el uso del desfibrilador.
\end{itemize}

\subsection{Principales funciones de un desfibrilador:}
\begin{itemize}
    \item Detectar arritmias cardíacas: El desfibrilador puede detectar ritmos cardíacos peligrosos súbitos o un paro cardíaco.
    \item Administrar descargas eléctricas: Si un desfibrilador detecta un paro cardíaco o una arritmia peligrosa, puede enviar una descarga eléctrica al corazón para tratar de recuperar un ritmo normal.
    \item Actuar como marcapasos: Algunos desfibriladores también actúan como marcapasos y administran terapia de estimulación para ayudar al corazón a latir a un ritmo normal.
    \item Monitorear el ritmo cardíaco: Los desfibriladores cardioversores implantables o portátiles monitorean los latidos del corazón todo el tiempo.
\end{itemize}

\subsection{Historia del desfibrilador:}
\begin{itemize}
    \item Desarrollo histórico y primeros usos.
    \item Evolución tecnológica de los desfibriladores.
\end{itemize}

La historia del desfibrilador se remonta a 1775, cuando el veterinario danés Abildgaard describió la utilización de corriente eléctrica para quitar la vida de una gallina y de su posterior recuperación del pulso, gracias a una descarga. En 1849, Ludwig y Hoffa, en Alemania, fueron los primeros en interpretar lo que Abildgaard había concluido sobre aquel hecho, y fueron ellos quienes dieron origen y definición al término Fibrilación Ventricular o FV. En 1947, Claude Beck, profesor de cirugía en la Universidad Case Western Reserve, realizó la primera desfibrilación con éxito en humanos, utilizando palas internas a ambos lados del corazón especialmente diseñadas para ello. Abrió quirúrgicamente el pecho de un muchacho de 14 años que se había quedado sin pulso, realizando masaje cardíaco manual durante 45 minutos hasta la llegada del desfibrilador. \newline \hfill \break
En 1960 se sustituyeron los primeros dispositivos de corriente alterna por los de corriente continua. Este último, al causar menos complicaciones, pareció inmediatamente más eficaz. En 1965, Frank Pantridge, profesor de Irlanda del Norte, inventó el primer desfibrilador portátil. Utilizaba un dispositivo alimentado por una batería de automóvil y se instaló en una ambulancia y se usó por primera vez en 1966. Posteriormente se introdujeron los primeros modelos de desfibrilador implantable. En caso de fibrilación necesaria, era capaz de dar una descarga de hasta 34 julios. Claramente, con el progreso tecnológico, estos dispositivos también se han mejorado.

\subsection{La física del desfibrilador:}

Los procesos físicos involucrados en el funcionamiento del instrumento son los siguientes.

\subsubsection{Energía potencial eléctrica:}
Una carga ejercerá una fuerza sobre cualquier otra carga y la energía potencial surge de un conjunto de cargas. Por ejemplo, si fijamos en cualquier punto del espacio una carga positiva Q, cualquier otra carga positiva que se traiga a su cercanía, experimentará una fuerza de repulsión y por lo tanto tendrá energía potencial.

\subsubsection{Potencial eléctrico:}

Se puede definir el potencial eléctrico como el trabajo realizado por una fuerza externa para llevar una carga de un punto a otro dentro de un campo electrostático en contra de la fuerza eléctrica, es importante resaltar que esto solo podría definirse para un campo estático producido por cargas que ocupan una región finita en el espacio. En el caso de que las cargas estén fuera del campo mencionado, entonces dichas cargas no cuentan con energía potencial y se considera que el potencial sería el trabajo necesario para llevar la carga hasta el interior del campo, al punto de considerarse. La unidad del sistema internacional para medir el potencial eléctrico es el voltio (V). \newline \hfill \break
El voltio se define de forma equivalente como la diferencia de potencial existente entre dos puntos tales que hay que realizar un trabajo de 1J para trasladar de un punto a otro la carga de 1C.

\subsubsection{Componentes del desfibrilador:}
\begin{itemize}
    \item Condensador: El componente más importante de un desfibrilador es un condensador que almacena una gran cantidad de energía en forma de carga eléctrica y luego la libera durante un corto período de tiempo. Un condensador consta de un par de conductores (por ejemplo, placas de metal) separados por un aislante (llamado dieléctrico). Los conductores pierden y ganan electrones con facilidad y, por lo tanto, permiten que fluya la corriente; mientras que los aislantes no pierden sus electrones y apenas permiten que fluya ninguna corriente.
    \item Dieléctricos: Son denominados con este nombre cuando son materiales que no son conductores. Los cuales tienen 3 funciones:
    \begin{enumerate}
        \item Resolver los problemas mecánicos de mantener dos hojas metálicas grandes con una separación muy pequeña sin generar contacto.
        \item Incrementar al máximo posible la diferencia de potencial entre las placas del capacitor.
        \item La capacitancia de un capacitor de dimensiones dadas es mayor cuando entre sus placas hay un material dieléctrico en lugar de vacío, esto puede ser observado por medio de un instrumento de medición de electrómetro sensible.
    \end{enumerate}
    \item Capacitadores: son los dos conductores separados por un espacio forman un capacitor la mayoría de las veces cada conductor tiene una carga neta 0, y los electrones son transferidos de un conductor a otro, a esta acción se le denomina cargar al capacitador. Cuando se dice que el capacitador tiene una carga Q o que está cargada Q está almacenando en el capacitor, su significado es que el conductor con el potencial más elevado es +Q y el conductor con bajo potencial será -Q.
    \item Inductores: son utilizados para prolongar la duración del flujo de corriente. Son bobinas de alambre que producen un campo magnético cuando la corriente fluye a través de ellas. Cuando la corriente pasa a través de un inductor, genera un flujo de electricidad en la dirección opuesta que se opone al flujo de corriente.
    \item Fuente de alimentación: Los transformadores elevadores se utilizan para convertir la tensión de red de 240 V CA a 5000 V CA. Luego, este se convierte a 5000 V CC mediante un rectificador. En la práctica, se usa un transformador elevador de voltaje variable para que el médico pueda seleccionar diferentes cantidades de carga.
\end{itemize}

El funcionamiento de un desfibrilador se basa en la generación, acumulación y liberación controlada de energía eléctrica para restablecer el ritmo cardíaco normal en casos de arritmias potencialmente mortales.\newline \hfill \break
En primer lugar, el desfibrilador cuenta con una fuente de alimentación que suministra corriente continua (CC). Cuando el interruptor del desfibrilador se encuentra en la posición 1, se permite que la corriente fluya desde la fuente de alimentación hacia un condensador presente en el dispositivo.\newline \hfill \break
El condensador es un componente crucial del desfibrilador, ya que almacena la energía eléctrica necesaria para la descarga. Cuando la corriente fluye desde la fuente de alimentación, comienza a acumularse carga en cada electrodo del condensador. La placa inferior del condensador se carga negativamente, mientras que la placa superior se carga positivamente.\newline \hfill \break
A medida que se acumula carga en las placas del condensador, se crea una diferencia de potencial entre ellas (V), la cual se opone a la fuerza electromagnética de la fuente de alimentación (E). Inicialmente, cuando no hay carga en las placas, la diferencia de potencial es cero y resulta fácil mover electrones hacia las placas. Sin embargo, a medida que la diferencia de potencial (V) aumenta, se opone al movimiento de electrones y se requiere más trabajo para seguir acumulando carga.\newline \hfill \break
El condensador continúa acumulando carga hasta que la diferencia de potencial entre las placas (V) alcanza el valor de la fuerza electromagnética de la fuente de alimentación (E). En este punto, la corriente deja de fluir y el condensador está completamente cargado. La cantidad de carga almacenada en el condensador se calcula como el producto de la capacitancia (C) del condensador y la diferencia de potencial (V) entre las placas.\newline \hfill \break
Cuando las paletas del desfibrilador se aplican al pecho del paciente y el interruptor se mueve a la posición 2, se completa un circuito. Los electrones almacenados en la placa inferior del condensador pueden pasar a través del paciente y regresar a la placa superior, lo que provoca que la corriente fluya y se libere la energía eléctrica almacenada.\newline \hfill \break
Durante este proceso de descarga, la diferencia de potencial entre las placas del condensador disminuye gradualmente a medida que la energía eléctrica se libera. La tasa de descarga disminuye exponencialmente a medida que la diferencia de potencial entre las placas disminuye, y este proceso está determinado por la capacitancia y la resistencia del circuito a través del cual fluye la corriente.\newline \hfill \break
En resumen, el desfibrilador utiliza un condensador para almacenar energía eléctrica, la cual se acumula y se libera de manera controlada para administrar una descarga eléctrica al corazón del paciente y restablecer el ritmo cardíaco normal en casos de arritmias cardíacas potencialmente mortales.

\newpage
\section{Experimento}

\textbf{"Circuito Pulsador" para modelar la acción de un desfibrilador}.
\subsection{Objetivo del experimento:}
Demostrar cómo un circuito puede generar un pulso eléctrico controlado en tiempo y magnitud, similar a como un desfibrilador administra una descarga para restablecer el ritmo cardíaco.
\subsection{Materiales:}
\begin{itemize}
    \item - Fuente de voltaje ajustable.
    \item Capacitor de alta capacidad.
    \item Bobina de inducción (inductor).
    \item Resistencias de varios valores.
    \item LED o pequeña lámpara de incandescencia.
    \item Cables de conexión.
    \item Protoboard o placa de montaje.
    \item Interruptor o pulsador.
    \item Multímetro.
\end{itemize}

\subsection{Procedimiento:}
\begin{itemize}
    \item Se construye un circuito simple en la protoboard que incluya la fuente de voltaje, el capacitor, la resistencia y el LED.
    \item La fuente de voltaje se ajusta para simular el voltaje de carga del desfibrilador.
    \item El capacitor se cargará al activar el circuito, y el LED permanecerá apagado debido a que la corriente no está fluyendo hacia él.
    \item Al cerrar el interruptor, el capacitor se descarga, y el LED se iluminará brevemente, simulando el pulso del desfibrilador.
    \item Se puede utilizar el multímetro para mostrar la caída de voltaje a través del capacitor durante la descarga.
\end{itemize}

\subsection{Justificación:}
Este experimento proporciona una visualización directa de cómo la energía almacenada en un capacitor puede liberarse de forma rápida y controlada. Además, se pueden cambiar resistencias para mostrar cómo esto afecta la rapidez y magnitud del pulso eléctrico, lo cual es fundamental en la operación de un desfibrilador para adaptarse a diferentes situaciones clínicas.

\subsection{Resultados Esperados:}
Los espectadores podrán ver cómo la energía almacenada en el capacitor es liberada en forma de pulso eléctrico y cómo los diferentes componentes del circuito afectan este proceso. Además, se promueve la comprensión de conceptos como el tiempo de carga y descarga, la relación entre corriente y voltaje, y la energía potencial eléctrica.\newline \hfill \break
Este experimento es relevante, educativo y seguro para llevarse a cabo en un entorno de clase, y aunque no es un desfibrilador real, ilustra muchos de los principios físicos que se utilizan en un equipo médico de esta naturaleza.

